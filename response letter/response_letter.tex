%%%%%%%%%%%%%%%%%%%%%%%%%%%%%%%%%%%%%%%%%%%%%%%%%%%%%
\documentclass{article}
%\documentclass[preprint2]{aastex61}
%\documentclass[12pt,preprint]{aastex}

\usepackage[table,usenames,dvipsnames]{xcolor}
\usepackage{amsmath}
\usepackage{subfigure}
\usepackage[backref,breaklinks,colorlinks,citecolor=blue]{hyperref}
\usepackage{natbib}
%\usepackage{natbib}
\bibliographystyle{fapj}
\usepackage{graphicx}
\usepackage{multirow}
\usepackage{soul}

%\newcommand{\jcap}{JCAP}

\newcommand{\sqdeg}{deg$^2$ }
\newcommand{\omb}{\ensuremath{\Omega_b h^2}}
\newcommand{\omc}{\ensuremath{\Omega_c h^2}}
\newcommand{\clpp}{\ensuremath{C_{L}^{\phi\phi}}}
\newcommand{\cpmf}{\ensuremath{C_{\ell}^{\rm PMF}}}

\newcommand{\cpmftens}{\ensuremath{C_{\ell}^{\rm PMF,\,tens}}}
\newcommand{\cpmfvec}{\ensuremath{C_{\ell}^{\rm PMF,\,vec}}}
\newcommand{\apmf}{\ensuremath{A_{\rm PMF}}}
\newcommand{\bpmf}{\ensuremath{B_{\rm 1\,Mpc}}}
\newcommand{\alens}{\ensuremath{A_{\rm lens}}}
\newcommand{\lcdm}{\ensuremath{\Lambda}CDM}
\newcommand{\nrun}{\ensuremath{n_{\rm run}}}
\newcommand{\neff}{\ensuremath{N_{\rm eff}}}
\newcommand{\ho}{H\ensuremath{_0}}
\newcommand{\mnu}{\ensuremath{\sum m_\nu}}
\newcommand{\ukarcmin}{\ensuremath{\mu}{\rm K-arcmin}}
\newcommand{\lknee}{\ensuremath{\ell_{\rm knee}}}
\newcommand{\lcut}{\ensuremath{\ell_{\rm min}}}
\newcommand{\fermilat}{\textit{Fermi}-LAT}

\newcommand{\be}{\begin{equation}}
\newcommand{\ee}{\end{equation}}
\newcommand{\planck}{{\sl Planck}}
\newcommand{\wmap}{{\sl WMAP}}
\newcommand{\bicepkeck}{BICEP2/Keck Array}
\newcommand{\sptnew}{SPT-3G }
\newcommand{\pb}{\textsc{Polarbear}}
\newcommand{\simons}{Simons Array}
\newcommand{\sptpol}{SPTpol}
\newcommand{\advactpol}{Adv.~ACTpol }

\newcommand{\tbd}[1]{\textcolor{Red}{{\bf TBD}: #1}}
\newcommand{\gab}[1]{\textcolor{Orchid}{[{\bf GS}: #1]}}
\newcommand{\changed}[1]{\textcolor{Red}{#1}}
\newcommand{\removed}[1]{\textcolor{Red}{}}
\newcommand{\question}[1]{\textcolor{Blue}{#1}}
\newcommand{\CR}[1]{\textit{\textcolor{Blue}{#1}}}
\newcommand{\CRremove}[1]{\st{#1}}
\include{number_list}

%

\usepackage{fullpage}
\newlength\tindent
\setlength{\tindent}{\parindent}
\setlength{\parindent}{0pt}
\renewcommand{\indent}{\hspace*{\tindent}}

% ref to section \S\ref{sec:label}

%\submitjournal{ApJ}
\def\Melbourne{1}
\def\uci{2}
%%%%%%%%%%%%%%%%%%%%%%%%%%%%%%%%%%%%%%%%%%%%%%%%%%%%%
\begin{document}

%\title{MNRAS: MN-18-3671-MJ\\Response to the Referee}

%\maketitle

\begin{center}
\Large MNRAS: MN-18-3671-MJ\\Response to the Referee
\end{center}

Thank you for taking the time to reassess our manuscript and review our response. We appreciate your comments with regards to the compression power of few-bit digitisation in conjunction with existing data reduction methods. Please find a detailed response to the points you raised below.

\vspace{0.5cm}

\hrule
\vspace{0.1cm}
\hrule

\vspace{0.5cm}

Reviewer's comment:

\indent \question{I am a bit shocked that the performance in conjunction with LPC-/PCM-based compression algorithms is as good as claimed here [...] My suspicion here is that the synthetic data under test have much higher entropy [...]}

\vspace{0.25cm}

Authors' reply:

\indent We appreciate the concern raised. To address this issue appropriately we followed the referee's suggestion and repeated the analysis using genuine experimental data.
We obtain SPT-3G data and use the FLAC compression algorithm used by the SPT. We apply few-bit digitisation to the TOD and apply FLAC. This allows us to compare the resulting file sizes between the few-bit data and the typical compression rates achieved for SPT-3G data.
Applying FLAC to few-bit timestreams does not lead to additional data volume reduction. We altered subsection 3.3 to reflect this insight.

\vspace{0.25cm}

Changes to the manuscript:

\indent \textbf{\changed{3.3 Existing Compression Techniques}}

\changed{Current-generation ground-based CMB experiments employ a combination of lossly, e.g. downsampling, and lossless compression, e.g. FLAC, bzip, LZMA, to manage their transmission bottlenecks. We test whether the data volume of timestreams that have undergone few-bit digitisation can be reduced further by applying the most popular lossless compression algorithm: FLAC. }

\changed{We obtain 2 minutes of SPT-3G TOD and apply the digitisation schemes in equations \ref{eq:1bit}-\ref{eq:3bit} before compressing the data further using FLAC. We compare the file size of FLAC compressed few-bit timestreams to the typical data reduction rate achieved for 24-bit SPT-3G data.}

\changed{We find that using FLAC in conjunction with extreme digitisation does not yield additional data volume reduction. For 3-bit digitisation, both the FLAC and raw timestream have $\sim$3 bits per data point. In fact, due to the overhead required by FLAC compressed files perform marginally worse than uncompressed few-bit streams. This is unsurprising, given that we have already maximally compressed the data using extreme digitisation. For comparison, FLAC reduces the full 24-bit  SPT-3G data to $6-8$ bits of entropy per number. For this real-world case, three-bit digitisation reduces the data volume by a factor of $2-3$.}

\vspace{0.5cm}

\hrule

\vspace{0.5cm}

Reviewer's comment:

\indent \question{[...] what is the impact on non-linearity in power spectrum estimation? [...]}

\vspace{0.25cm}

Authors' reply:

\indent Thank you for raising this point. Indeed the non-linearity introduced by few-bit digitisation disappears in the power spectrum estimation. This is because many TOD points contribute to each map pixel and hence the central limit theorem ensures that the distribution within each pixel approaches a Gaussian. We appended a sentence stating this to subsection 2.4 (Power Spectrum Estimation).

\vspace{0.25cm}

Changes to the manuscript:

\indent \changed{Non-linearity introduced by the digitisation process disappears in the final map due to the large number of TOD points that contribute to each map pixel. Thus the digitisation effects are not an issue for power spectrum estimation.}


\end{document}

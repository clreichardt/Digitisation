%%%%%%%%%%%%%%%%%%%%%%%%%%%%%%%%%%%%%%%%%%%%%%%%%%%%%
\documentclass[apj]{emulateapj}
%\documentclass[preprint2]{aastex61}
%\documentclass[12pt,preprint]{aastex}
\graphicspath{{figures/}}
\DeclareGraphicsExtensions{.jpg,.pdf,.png,.eps,.ps}

\usepackage[table,usenames,dvipsnames]{xcolor}
%\usepackage{amsmath}
%\usepackage{subfigure}
\usepackage[backref,breaklinks,colorlinks,citecolor=blue]{hyperref}
\usepackage{natbib}
%\usepackage{natbib}
\bibliographystyle{fapj}
%\usepackage{graphicx}
%\usepackage{multirow}
\usepackage{soul}

%\newcommand{\jcap}{JCAP}

\newcommand{\sqdeg}{deg$^2$ }
\newcommand{\omb}{\ensuremath{\Omega_b h^2}}
\newcommand{\omc}{\ensuremath{\Omega_c h^2}}
\newcommand{\clpp}{\ensuremath{C_{L}^{\phi\phi}}}
\newcommand{\cpmf}{\ensuremath{C_{\ell}^{\rm PMF}}}

\newcommand{\cpmftens}{\ensuremath{C_{\ell}^{\rm PMF,\,tens}}}
\newcommand{\cpmfvec}{\ensuremath{C_{\ell}^{\rm PMF,\,vec}}}
\newcommand{\apmf}{\ensuremath{A_{\rm PMF}}}
\newcommand{\bpmf}{\ensuremath{B_{\rm 1\,Mpc}}}
\newcommand{\alens}{\ensuremath{A_{\rm lens}}}
\newcommand{\lcdm}{\ensuremath{\Lambda}CDM}
\newcommand{\nrun}{\ensuremath{n_{\rm run}}}
\newcommand{\neff}{\ensuremath{N_{\rm eff}}}
\newcommand{\ho}{H\ensuremath{_0}}
\newcommand{\mnu}{\ensuremath{\sum m_\nu}}
\newcommand{\ukarcmin}{\ensuremath{\mu}{\rm K-arcmin}}
\newcommand{\lknee}{\ensuremath{\ell_{\rm knee}}}
\newcommand{\fermilat}{\textit{Fermi}-LAT}

\newcommand{\be}{\begin{equation}}
\newcommand{\ee}{\end{equation}}
\newcommand{\planck}{{\sl Planck}}
\newcommand{\wmap}{{\sl WMAP}}
\newcommand{\bicepkeck}{BICEP2/Keck Array}
\newcommand{\sptnew}{SPT-3G}
\newcommand{\pb}{\textsc{Polarbear}}
\newcommand{\simons}{Simons Array}
\newcommand{\sptpol}{SPTpol}
\newcommand{\advactpol}{Adv.~ACTpol}

\newcommand{\tbd}[1]{\textcolor{Red}{{\bf TBD}: #1}}
\newcommand{\gab}[1]{\textcolor{Orchid}{[{\bf GS}: #1]}}
\newcommand{\changed}[1]{\textcolor{Red}{#1}}
\newcommand{\removed}[1]{\textcolor{Red}{}}
\include{number_list}

%

% ref to section \S\ref{sec:label}

%\submitjournal{ApJ}
\def\Melbourne{1}
\def\uci{2}
%%%%%%%%%%%%%%%%%%%%%%%%%%%%%%%%%%%%%%%%%%%%%%%%%%%%%
\begin{document}

\title{Digitisation}
\author{L.~Balkenhol\altaffilmark{\Melbourne} and C.~L.~Reichardt\altaffilmark{\Melbourne}}
\altaffiltext{\Melbourne}{School of Physics, University of Melbourne, Parkville, VIC 3010, Australia}
\email{christian.reichardt@unimelb.edu.au}

\begin{abstract}

It's awesome!
\end{abstract}

\keywords{ cosmic background radiation --- polarization }
\section{Introduction}
\label{sec:intro}

CMB is great; One reason is that there is a history of compression and computational techniques that reduce the load of large datasets. ie maps; bandpowers; pseudo-cls.

satellites have also reduced bits on TOD; ground based haven't had to yet

however as we discuss building ever larger arrays at remote sites, we are starting to be limited: spt example.

in this work we present digitisation for ground-based cmb polarisation measurements.
teaser results


The outline of this paper is as follows. 
We present the digisation schemes in \S\ref{sec:dig}, and their performance in \S\ref{sec:results}
We summarize our findings in \S\ref{sec:conclusions}. 

\section{Digitisation}
\label{sec:dig}

overview

\subsection{Problem}
\label{subsec:problem}

why is it worth considering



\subsection{Methods}
\label{subsec:method}

\begin{figure*}[htb]\centering
\includegraphics[width=0.9\textwidth,clip,trim={1.5cm 12.5cm 5cm 3.8cm}]{pretty.pdf}
  \caption[Current ]{
  Current 
           \label{fig:ig}
  }
\end{figure*}






\section{Results}
\label{sec:results}

\begin{table}[tbh]
\begin{center}
\caption{\label{tab:noise} Noise levels}
\small
\begin{tabular}{l | c c c }
Model   & XX&XX&$r$\\
\hline

\end{tabular}
\tablecomments{ 
go go
} \normalsize
\end{center}
\end{table}

\begin{table*}[tbh]
\begin{center}
\caption{\label{tab:experiments} Assumed survey parameters}
\small
\begin{tabular}{l || c c c c c }
Experiment & Sky coverage & Polarized Noise level  & 1/$f$ knee & Beam FWHM \\
& &($\mu$K-arcmin)&&(arcmin.)\\
\hline
\tiny \\ \small
CMB Stage III & & & & \\
~~~~~SPT-3G & 6\% & 3.0 & 200 & 1.2 \\
~~~~~Simons Array & 36\% & 9.5 & 200 & 3.5 \\ 
\tiny \\ \small
%\hline
CMB Stage IV & 55\% & 1.3 & 100 & 4.0 \\
\end{tabular}
\tablecomments{ 
Key numbers about the planned stage III and IV experiments. 
The sky coverage percentages are after galactic cuts. 
Unless otherwise noted,  the Fisher matrix forecasts in this work use these numbers. 
All forecasts also allow for beam and calibration uncertainties as noted in the text. 
} \normalsize
\end{center}
\end{table*}

\section{Conclusions}
\label{sec:conclusions}
\acknowledgments

We thank the \changed{referee as well as} Srinivasan Raghunathan and Federico Bianchini for valuable feedback on the manuscript. 
We acknowledge support from an Australian Research Council Future Fellowship (FT150100074), and also from the University of Melbourne. 
This research used resources of the National Energy Research Scientific Computing Center, which is supported by the Office of Science of the U.S. Department of Energy under Contract No. DE-AC02-05CH11231. 
We acknowledge the use of the Legacy Archive for Microwave Background Data Analysis (LAMBDA). Support for LAMBDA is provided by the NASA Office of Space Science.


\bibliography{digitisation}


\end{document}
